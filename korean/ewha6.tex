\documentclass[a4paper, 11pt, oneside]{book}

\usepackage{kotex}
\usepackage{blindtext}
\usepackage{fontspec}
\usepackage{setspace}
\doublespacing
%\setmainhangulfont{NotoSansKR-Regular.otf}

\begin{document}
\setcounter{secnumdepth}{0}
\begin{titlepage}
\title{\Huge 이화 한국어 6 노트}
\author{\Large 고한별}
\date{}
\maketitle
\end{titlepage}

\pagebreak

\chapter{개인과 사회}
\section{문법}

\subsection{1: -(으)ㄴ/는 감이 있다}
\paragraph{뜻: 어떤 생각을 하고 있지만 확실하게 단정하지는 않음을 나타낸다.}
\paragraph{예문:}
\begin{itemize}
	\item 긴장하지 않았으면 좋았\textbf{을련마는} 긴장한 탓인지 실수를 많이 한 것 같아요 
	\item 저 카페는 미나 씨가 제일 좋아하는 곳\textbf{이련만} 요즘 돈이 없어서 못 가고 있어요.
	\item 면접 준비를 열심히 했으면 취직했\textbf{으련만} 시간 관리를 더 배워야겠네요. 
	\item 우리 국가의 번영을 위해 외교적 영향력을 키우는 프로그램을 제공해준다면 좋\textbf{으련만} 우리 정부는 국외에 대한 관심이 없어요.   
\end{itemize}

\subsection{2: -(으)련마는 [-(으)ㄹ 텐데]}
\paragraph{뜻: 어떤 조건을 충족된(satisfy) 상황을 기대하며 가정하지만(assume) 실제로는 그렇지 않음을 나타낸다. Kind of old and literary}
\paragraph{예문:}
\begin{itemize}
	\item 긴장하지 않았으면 좋았\textbf{을련마는} 긴장한 탓인지 실수를 많이 한 것 같아요 
	\item 저 카페는 미나 씨가 제일 좋아하는 곳\textbf{이련만} 요즘 돈이 없어서 못 가고 있어요.
	\item 면접 준비를 열심히 했으면 취직했\textbf{으련만} 시간 관리를 더 배워야겠네요. 
	\item 우리 국가의 번영을 위해 외교적 영향력을 키우는 프로그램을 제공해준다면 좋\textbf{으련만} 우리 정부는 국외에 대한 관심이 없어요.   
\end{itemize}



\end{document}

