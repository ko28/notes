\documentclass[a4paper]{article}
\usepackage[utf8]{inputenc}
\usepackage[margin=1in]{geometry}
%\usepackage[pdfborder={0 0 0},colorlinks=true,citecolor=red{}]{hyperref}
\usepackage{amsmath}
\usepackage{setspace}

\title{Math 341: Linear Alegbra}
\author{Daniel Ko}
\date{Spring 2020}

\doublespacing
\begin{document}
\maketitle


\section{Propositional Logic}

System of figuring out if something is true or false.
\\
$\text{Proposition} \rightarrow \text{Statement} \rightarrow \text{True or False}$

Examples:
\begin{itemize}
	\item P: Today is sunny 
	\item P: I'm 5'11
\end{itemize}

\subsection{We can compose them}
P: Today is sunny
\\
Q: Today is rainy
\\
$P \lor Q \implies \hbox{Today is sunny or Today is cloudy}$

\subsection{Connectors (functions on propositions)}

\subsubsection{Negation ($\lnot$)}

P: Today is sunny
\\
$\lnot P$: Today is not sunny

\begin{center}
	Truth Table
	\begin{displaymath}
		\begin{array}{|c|c|}
			\hline
			P & \lnot Q \\ 
			\hline
			T & F \\
			F & T \\
			\hline
		\end{array}
	\end{displaymath}
\end{center}


\subsubsection{Or ($\lor$)}
\begin{center}
	\begin{displaymath}
		\begin{array}{|c|c|c|}
			\hline
			P & Q & P \lor Q \\ 
			\hline
			T & T & T\\
			T & F & T\\
			F & T & T\\
			F & F & F\\ 
			\hline
		\end{array}
	\end{displaymath}
\end{center}

\subsubsection{And ($\land$)}
\begin{center}
	\begin{displaymath}
		\begin{array}{|c|c|c|}
			\hline
			P & Q & P \land Q \\ 
			\hline
			T & T & T\\
			T & F & F\\
			F & T & F\\
			F & F & T\\ 
			\hline
		\end{array}
	\end{displaymath}
\end{center}

\subsection{Implication}
$P \implies Q$ means P implies Q 
\\
In other words: if P, then Q is true.
\\
False can imply anything.
\\
We will come back to this.
\begin{center}
	\begin{displaymath}
		\begin{array}{|c|c|c|}
			\hline
			P & Q & P \implies Q \\ 
			\hline
			T & T & T\\
			T & F & F\\
			F & T & T\\
			F & F & T\\
			\hline
		\end{array}
	\end{displaymath}
\end{center}

\subsection{Equivalence}
$P \Leftrightarrow Q$	
\\
Means they have same truth value on a truth table.
\\
If you break this down to implications you get:
\\
$[(P \implies Q) \land (Q \implies P)] \Leftrightarrow [P \Leftrightarrow Q]$
\begin{center}
	\begin{displaymath}
		\begin{array}{|c|c|c|}
			\hline
			P & Q & P \Leftrightarrow Q \\ 
			\hline
			T & T & T\\
			T & F & F\\
			F & T & F\\
			F & F & T\\
			\hline
		\end{array}
	\end{displaymath}
\end{center}


($P \implies Q) \Leftrightarrow (\lnot P \lor Q)$ 
\begin{center}
	\begin{displaymath}
		\begin{array}{|c|c|c|c|c|}
			\hline
			P & \lnot P & Q & P \implies Q & \lnot P \lor Q\\ 
			\hline
			T & F & T & T & T\\
			T & F & F & F & F\\
			F & T & T & T & T\\
			F & T & F & T & T\\
			\hline
		\end{array}
	\end{displaymath}
\end{center}

\subsection{Rules for Computing}
\subsubsection{De Morgan's Laws}
$\lnot (P \lor Q) \Leftrightarrow (\lnot P) \land (\lnot Q)$
\\
The rules can be expressed in English as:
\begin{itemize}
	\item the negation of a disjunction is the conjunction of the negations
	\item the negation of a conjunction is the disjunction of the negations
\end{itemize}
Proof of De Morgan's law using proof table
\begin{center}
	\begin{displaymath}
		\begin{array}{|c|c|c|c|c|c|c|}
			\hline
			P & Q & \lnot (P \lor Q) & \lnot P \land \lnot Q & \lnot P & \lnot Q\\ 
			\hline
			T & T & F & F & F & F\\
			T & F & F & F & F & T\\
			F & T & F & F & T & F\\
			F & F & T & T & T & T\\
			\hline
		\end{array}
	\end{displaymath}
\end{center}

\subsubsection{Transitivity}
$[(P \implies R) \land (R \implies Q)] \implies (P \implies Q)$

\begin{center}
	\begin{displaymath}
		\begin{array}{|c|c|c|c|c|c|c|}
			\hline
			P & Q & R & P \implies R & R \implies P & P \implies R \land R \implies P &\ 
			\hline
			T & T & T & F & F & F\\
			T & T & F & F & F & T\\
			T & F & T & F & T & F\\
			T & F & F & T & T & T\\
			F & T & T & T & T & T\\
			F & T & F & T & T & T\\
			F & F & T & T & T & T\\
			F & F & F & T & T & T\\
			\hline
		\end{array}
	\end{displaymath}
\end{center}




\end{document}

