\documentclass[a4paper]{article}
\usepackage{cmbright}
\usepackage[T1]{fontenc}
\usepackage{enumerate}
\usepackage[utf8]{inputenc}
\usepackage[margin=1in]{geometry}
%\usepackage[pdfborder={0 0 0},colorlinks=true,citecolor=red{}]{hyperref}
\usepackage{amsmath}
\usepackage{amssymb}
\usepackage{setspace}

\title{Math 341: Linear Alegbra}
\author{Daniel Ko}
\date{Spring 2020}

\doublespacing
\begin{document}
\maketitle


\section{Propositional Logic}

System of figuring out if something is true or false.
\\
$\text{Proposition} \rightarrow \text{Statement} \rightarrow \text{True or False}$

Examples:
\begin{itemize}
	\item P: Today is sunny 
	\item P: I'm 5'11
\end{itemize}

\subsection{We can compose them}
P: Today is sunny
\\
Q: Today is rainy
\\
$P \lor Q \Rightarrow \hbox{Today is sunny or Today is cloudy}$

\subsection{Connectors (functions on propositions)}

\subsubsection{Negation ($\lnot$)}

P: Today is sunny
\\
$\lnot P$: Today is not sunny

\begin{center}
	Truth Table
	\begin{displaymath}
		\begin{array}{|c|c|}
			\hline
			P & \lnot Q \\ 
			\hline
			T & F \\
			F & T \\
			\hline
		\end{array}
	\end{displaymath}
\end{center}


\subsubsection{Or ($\lor$)}
\begin{center}
	\begin{displaymath}
		\begin{array}{|c|c|c|}
			\hline
			P & Q & P \lor Q \\ 
			\hline
			T & T & T\\
			T & F & T\\
			F & T & T\\
			F & F & F\\ 
			\hline
		\end{array}
	\end{displaymath}
\end{center}

\subsubsection{And ($\land$)}
\begin{center}
	\begin{displaymath}
		\begin{array}{|c|c|c|}
			\hline
			P & Q & P \land Q \\ 
			\hline
			T & T & T\\
			T & F & F\\
			F & T & F\\
			F & F & T\\ 
			\hline
		\end{array}
	\end{displaymath}
\end{center}

\subsection{Implication}
$P \Rightarrow Q$ means P implies Q 
\\
In other words: if P, then Q is true.
\\
False can imply anything.
\\
We will come back to this.
\begin{center}
	\begin{displaymath}
		\begin{array}{|c|c|c|}
			\hline
			P & Q & P \Rightarrow Q \\ 
			\hline
			T & T & T\\
			T & F & F\\
			F & T & T\\
			F & F & T\\
			\hline
		\end{array}
	\end{displaymath}
\end{center}

\subsection{Equivalence}
$P \Leftrightarrow Q$	
\\
Means they have same truth value on a truth table.
\\
If you break this down to implications you get:
\\
$[(P \Rightarrow Q) \land (Q \Rightarrow P)] \Leftrightarrow [P \Leftrightarrow Q]$
\begin{center}
	\begin{displaymath}
		\begin{array}{|c|c|c|}
			\hline
			P & Q & P \Leftrightarrow Q \\ 
			\hline
			T & T & T\\
			T & F & F\\
			F & T & F\\
			F & F & T\\
			\hline
		\end{array}
	\end{displaymath}
\end{center}


($P \Rightarrow Q) \Leftrightarrow (\lnot P \lor Q)$ 
\begin{center}
	\begin{displaymath}
		\begin{array}{|c|c|c|c|c|}
			\hline
			P & \lnot P & Q & P \Rightarrow Q & \lnot P \lor Q\\ 
			\hline
			T & F & T & T & T\\
			T & F & F & F & F\\
			F & T & T & T & T\\
			F & T & F & T & T\\
			\hline
		\end{array}
	\end{displaymath}
\end{center}

\subsection{Rules for Computing}

\subsubsection{Distibutive Laws}
$[P \land (Q \lor R)]\Leftrightarrow[(P \land Q) \lor (P \land R)]$\\
$[P \lor (Q \land R)]\Leftrightarrow[(P \lor Q) \land (P \lor R)]$

\subsubsection{Associative Laws}
$X \lor (Y \lor Z) \Leftrightarrow (X \lor Y) \lor Z $\\
$X \land (Y \land Z) \Leftrightarrow (X \land Y) \land Z $

\subsubsection{De Morgan's Laws}
$\lnot (P \lor Q) \Leftrightarrow (\lnot P) \land (\lnot Q)$
\\
The rules can be expressed in English as:
\begin{itemize}
	\item the negation of a disjunction is the conjunction of the negations
	\item the negation of a conjunction is the disjunction of the negations
\end{itemize}
Proof of De Morgan's law using proof table
\begin{center}
	\begin{displaymath}
		\begin{array}{|c|c|c|c|c|c|c|}
			\hline
			P & Q & \lnot (P \lor Q) & \lnot P \land \lnot Q & \lnot P & \lnot Q\\ 
			\hline
			T & T & F & F & F & F\\
			T & F & F & F & F & T\\
			F & T & F & F & T & F\\
			F & F & T & T & T & T\\
			\hline
		\end{array}
	\end{displaymath}
\end{center}

\subsubsection{Transitivity}
$[(P \Rightarrow R) \land (R \Rightarrow Q)] \Rightarrow (P \Rightarrow Q)$

\begin{center}
	\begin{displaymath}
		\begin{array}{|c|c|c|c|c|c|c|}
			\hline
			P & Q & R & P \Rightarrow R & R \Rightarrow P & P \Rightarrow R \land R \Rightarrow P & P \Rightarrow Q \\ 
			\hline
			T & T & T & T & T & T\\
			T & T & F & F & T & F\\
			T & F & T & T & T & T\\
			T & F & F & F & T & F\\
			F & T & T & T & F & F\\
			F & T & F & T & F & F\\
			F & F & T & T & F & F\\
			F & F & F & T & F & F\\
			\hline
		\end{array}
	\end{displaymath}
\end{center}
I'll do this later lol
\subsection{Notation}
$\forall
\\
\exists
\setminus
\exists \!
$
\subsection{Logical Concepts}
\subsubsection{Logical Truth}
Logical truth, sometimes called tautology, means a proposition is true in all possible cases. \\
For example: \ $A \lor \lnot A$ \ is always true. 
\begin{center}
	\begin{displaymath}
		\begin{array}{|c|c|c|}
			\hline
			A & \lnot A & A \lor \lnot A \\ 
			\hline
			T & F & T\\
			F & T & T\\
			\hline
		\end{array}
	\end{displaymath}
\end{center}

\subsubsection{Logical Contradiction}
Similar to a logical truth, a logical condtradiction means a proposition is false in all possible cases
For example: \ $A \land \lnot A$ \ is always false. 
\begin{center}
	\begin{displaymath}
		\begin{array}{|c|c|c|}
			\hline
			A & \lnot A & A \land \lnot A \\ 
			\hline
			T & F & F\\
			F & T & F\\
			\hline
		\end{array}
	\end{displaymath}
\end{center}

\subsubsection{Law of Logically True Conjunct}
If $Y$ is a logical truth, then $X \land Y \Leftrightarrow X$

\subsubsection{Law of Contradictory Disjunct}
If $Y$ is a contradiction, then $X \lor Y \Leftrightarrow X$

\subsubsection{Disjunctive Normal Form}
Forumla consisting of disjunction of conjunctions, described as an $\lor$ of $\land$\\
$A \lor \lnot B \Leftrightarrow (A \land B)\lor(A \land \lnot B)\lor(\lnot A \land \lnot B)$\\
Beyond the scope of this class.
\subsubsection{Expressive Completeness}
A connective, or set of connectives is expressively complete iff every truth function can be represented just using the connective or connectives.\\
Example is sheffer stroke, also known as NAND\\ 
Simpler example is $\lnot$ and $\land$.\\
Beyond the scope of this class.
\subsubsection{Logically Valid vs Logically Sound}
Logically valid means if the premises are true, the conclusion must be true.
In other words, an argument is logically valid iff it takes a form that makes it impossible for the premises to be true and the conclusion to be false.
Doesn't mean the argument is actually true, just its structure.\\
An argument is logically sound if the premises are true and is logically valid.

\subsection{Proof Techniques (basic ones)}

\begin{enumerate}[i]
	\item $(P \Leftrightarrow Q) \Leftrightarrow (P \Rightarrow Q) \land (Q \Rightarrow P)$ 
	\item $(P \Rightarrow R) \land (R \Rightarrow Q) \Rightarrow (P \Rightarrow Q)$
	\item $(P \lor Q) \Rightarrow R \Leftrightarrow (P \Rightarrow R) \land (Q \Rightarrow R)$\\ Note: false can imply anything
	\item{$(P \Rightarrow Q) \Leftrightarrow (\not Q \Rightarrow \not P)$}
	\item {$[(P \Rightarrow Q) \Leftrightarrow True] \Leftrightarrow [\lnot (P \Rightarrow Q) \Leftrightarrow False]$}
\end{enumerate}



\section{Vector space}
\subsection{Field}
Def: A field F is a set with two operations (+, $\cdot$ ) satisfying

\noindent $\forall\ x,y \in F$\\
\-\hspace{1in}$\exists !\ z \in F\ s.t.\ z = x + y$\\
\-\hspace{1in}$\exists !\ w \in F\ s.t.\ w = x \cdot y$\\ 
This is called closure.
\noindent Other properties:\\
$\forall\ a,b,c \in F$
\begin{enumerate}
	\item{a + b = b + a}
	\item{(a + b) + c = a + (b + c)}
	\item{$\exists \ 0 \in F,\ \exists \ 1 \in F$}\\
		$0 + a = a,\ 1 \cdot a = a$ 
	\item{Additive and Multiplicative inverse \\ $\forall \ a \in F,\ \forall \ b \in F \setminus \{0\} \\ \exists \ c,d \in F \ s.t. \ a + b = 0, \ bd = 1$}
	\item{$a \cdot (b + c) = ab + ac$}
\end{enumerate}

Examples of fields: $\mathbb{R},\ \mathbb{C}$

\subsection{Vector space}
Def: A vector space V over a field F is a set with two operations
\begin{itemize}
	\item{addition}
	\item {scalar multiplication}
\end{itemize}
which satisfies
\begin{enumerate}
	\item{$\forall \ x,y \in V \\ x + y = y + x$}
	\item{$\forall \ x,y,z \in V \\ (x + y) + z = x + (y + x)$}
	\item{$\exists \ 0 \in V s.t. \ x + 0 = x $}
	\item{$\forall \ x \in V\ \exists \ y \in V\ s.t. \ x + y = 0$}
	\item{$\forall \ x \in V \ s.t. \ 1 \cdot x = x$ \ (1 from field F)}
	\item{$\forall \ a,b \in F, \  \forall \ x \in V \\ (a \cdot b) \cdot x = a \cdot ( b \cdot x)$}
	\item{$\forall \ a \in F, \ \exists \ x,y \in V \\ a(x + y) = ax + ay$}
	\item{$\forall \ a,b \in F, \ \forall \ x \in V \\ (a + b)x = ax + bx $}
	\item{$\forall \ x,y \in V\ \exists ! \ z \in V\ s.t. \ x + y = z$}
	\item{$\forall \ x \in F,\ \forall \ x \in V \ \exists ! \ w \ s.t. \ w = x + y$}
\end{enumerate}

$\Rightarrow$ \ Elements of F are are called scalars

\subsection{Example of a vector space: tuples of scalars}
An n-tuple is a sequence (or ordered list of n elements, aka order matters), where n is a non-negative integer. The set of all n-tuples with entries from a field $F$ is denoted by $F^n$\\
$F^n = \{(a_1, a_2, a_3, \dots, a_n) \ a_1 \in F\}$
\subsubsection{Adding n-tuples}
$u = (a_1, a_2,\dots, a_n) \ \ v = (b_1, b_2,\dots, b_n) $\\
$u + v = (a_1 + b_1, a_2 + b_2, a_3 + b_3, \dots, a_n + b_n)$
\subsubsection{Multiplying n-tuples with a scalar}
$c \in F$\\
$c \cdot u = (ca_1, ca_2,\dots, ca_n) $

\subsection{Another example of vector space: Matrices}
$M_{mxn}(F)$\ is the set of matrices with element in F of dimensions m x n\\
Generic matrix in $M$ where $a \in F$:
\begin{equation*}
A_{m,n} = 
\begin{pmatrix}
a_{1,1} & a_{1,2} & \cdots & a_{1,n} \\
a_{2,1} & a_{2,2} & \cdots & a_{2,n} \\
\vdots  & \vdots  & \ddots & \vdots  \\
a_{m,1} & a_{m,2} & \cdots & a_{m,n} 
\end{pmatrix}
\end{equation*}

\subsubsection{Caveat: How do we define $A = B$?}
\begin{equation*}
\begin{split}
	A = B \Leftrightarrow A_{i,j} = B_{i,j} \ \text{for} \ & 1 \leq i \leq m\\
													   & 1 \leq j \leq n
\end{split}
\end{equation*}

\subsubsection{Rules for matrices}
Addition: $(A + B)_{i,j} = A_{i,j} + B_{i,j}$\\
Multiplication by scalar: $(cA)_{i,j} = cA_{i,j}$


\end{document} 
